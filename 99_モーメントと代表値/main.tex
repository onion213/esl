\documentclass{jsarticle}
\usepackage[dvipdfmx]{graphicx}
\usepackage[dvipdfmx]{hyperref}
\usepackage{pxjahyper}
\usepackage{amsmath}
\usepackage{amsthm}
\usepackage{amsfonts}

\theoremstyle{definition}
\newtheorem*{theorem}{定理}
\newtheorem*{point}{嬉しいポイント}
\newtheorem*{definition}{定義}
\newtheorem*{claim}{主張}

\newcommand{\argmin}{\mathop{\mathrm{argmin}}\limits}
\newcommand{\argmax}{\mathop{\mathrm{argmax}}\limits}

\title{モーメントと代表値}

\author{鬼頭幸助}
\date{\today}
\begin{document}
\maketitle

実数のデータ$x_1,\dots,x_n$の代表値として有名なのは、平均、中央値、最頻値の3種類。
これらの統一的な理解と、これらを「データを代表する値」として採用することの正当化について。

\begin{claim}
平均、中央値、最頻値は、それぞれ、2次、1次、0次の中心化モーメントを最小化する値である。
\end{claim}

\begin{point}
  \begin{enumerate}
    \item 平均、中央値、最頻値という、一見全然別物だが、横並びで出てきがちな概念を統一的に理解できる。
    \item 外れ値に対する強さ(頑健性、robustness)の違いを、「重みの付け方」という観点で理解できる。

    モーメントの$\sum$の中を損失関数だと思うと、次数が高いものほど、外れ値に大きなペナルティを課していることが分かる。逆に、次数が低いものほど、外れ値を軽視もしくは無視するように働く。これが、「頑健さ」を生んでいる。
  \end{enumerate}
\end{point}

\begin{definition}
  実数データ$x_1,\dots,x_n$と$c \in \mathcal{R}$及び$p>0$に対して、点$c$における$p$次中心化モーメントとは、以下で定まる実数$\mu_p(c)$。
  \[
    \mu_p(c)=\sum_{i=1}^n|x_i-c|^p
  \]
  また、点$c$における$0$次中心化モーメント$\mu_0(c)$を以下によって定める。
  \[
    \mu_0(c)=\sum_{i=1}^n1-\delta_{x_i,c}
  \]
  ただし、$\delta_{i,j}$はKroneckerのデルタである。これは、$p>0$のときの定義において、$p \to 0$としたときの極限である。
\end{definition}

\begin{theorem}
  平均$\frac{1}{n}\sum_{i=1}^nx_i$は、2次中心化モーメントを最小化する。すなわち、
  \[
    \argmin_c\mu_2(c)=\frac{1}{n}\sum_{i=1}^nx_i
  \]
\end{theorem}
\begin{proof}
  2次中心化モーメントを、中心の値の関数とみると、微分可能なので、微分すればよい。
  \[
    \mu_2(c)=\sum_{i=1}^n(x_i-c)^2
  \]
  となるが、これは$c$に関する2次式で、$c^2$の係数は$n>0$なので、最小値を持つ。これを最小化する点を$\bar{x}$とすると、
  \[
    \frac{d}{dc}\mu_2(c)=\sum_{1=1}^n2(x_i-\bar{x})=0
  \]
  となるので、$\bar{x}$について解いて、
  \[
    \bar{x}=\sum_{i=1}^nx_i
  \]
  を得る。
\end{proof}

\begin{theorem}
  中央値$\mathrm{median}(\{x_i \mid i=1,\dots,n\})$は、1次中心化モーメントを最小化する。すなわち、
  \begin{align*}
    \argmin_c\mu_1(c)&=\mathrm{median}(\{x_i \mid i=1,\dots,n\})\\
    &=
    \begin{cases}
      x_{(\frac{n+1}{2})} & (n\mbox{が奇数のとき})\\
      \frac{1}{2}(x_{(\frac{n}{2})}+x_{(\frac{n}{2}+1)}) & (n\mbox{が偶数のとき})
    \end{cases}
  \end{align*}
  となる。ただし、$x_{(i)}$は、$x_1,\dots,x_n$を昇順に並べた時の$i$番目の値とする。
\end{theorem}

\begin{proof}
  $0 \le x_1 \le x_2 \le \cdots \le x_n$としてよい。

  $x_k \le c \le x_{k+1}$のときを考える。このとき、
  \begin{align*}
    \mu_1(c)&=\sum_{i=1}^n|x_i-c|\\
    &=\sum_{i=1}^k(c-x_i)+\sum_{i=k+1}^n(x_i-c)\\
    &=(2k-n)c-\sum_{i=1}^kx_i+\sum_{i=k+1}^nx_i
  \end{align*}
  となる。よって、
  \[
    \argmin_{x_k \le c \le x_{k+1}}=
    \begin{cases}
      x_k & (2k-n>0\mbox{のとき})\\
      x_{k+1} & (2k-n<0\mbox{のとき})\\
      \mbox{任意の}c & (2k-n=0\mbox{のとき})
    \end{cases}
  \]
  と分かる。よって、$n$が奇数のとき、$2k<n$ならば、
  \[
    \mu_1(c) \ge \mu_1(x_{k+1}) \ge \mu_1(x_{k+1}) \ge \cdots \ge \mu_1 (x_{\frac{n+1}{2}})
  \]
  となり、$2k>n$ならば、
  \[
    \mu_1(c) \ge \mu_1(x_{k}) \ge \mu_1(x_{k-1}) \ge \cdots \ge \mu_1 (x_{\frac{n+1}{2}})
  \]
  となる。よって、$n$が奇数のとき、
  \[
    \argmin_c\mu_1(c)=x_{\frac{n+1}{2}}=\mathrm{median}(\{x_i \mid i=1,\dots,n\})
  \]
  と分かる。また、$n$が偶数のとき、奇数のときと同様にして、
  \[
    \begin{cases}
      \mu_1(c) \ge \mu_1(x_{\frac{n}{2}}) & (2k<n\mbox{のとき})\\
      \mu_1(c) \ge \mu_1(x_{\frac{n}{2}+1}) & (2k>n\mbox{のとき})\\
    \end{cases}
  \]
  と分かり、$2k=n$のとき、$\mu_1(c)$は定数になるので、
  \[
    \argmin_c\mu_1(c)=(x_{\frac{n}{2}} \le c \le x_{\frac{n}{2}+1}\mbox{を満たす任意の値})
  \]
  となるが、もちろん中央値$\frac{1}{2}(x_{(\frac{n}{2})}+x_{(\frac{n}{2}+1)})$はこの区間に含まれる。

\end{proof}

\begin{theorem}
  最頻値$\mathrm{mode}(\{x_i \mid i=1,\dots,n\})$は、0次中心化モーメントを最小化する。すなわち、
  \begin{align*}
    \argmin_c\mu_0(c)&=\mathrm{mode}(\{x_i \mid i=1,\dots,n\})\\
    &=\argmax_c\#\{i \mid x_i=c\}
  \end{align*}
  ただし、$\#$は、集合の要素の数を表す。
\end{theorem}

\begin{proof}
  \begin{align*}
    \mu_0(c)&=\sum_{i=1}^n1-\delta_{x_i,c}\\
    &=\#\{i \mid x_i \ne c\}
  \end{align*}
  となるので、
  \begin{align*}
    \argmin_c\mu_0(c)&=\argmin_c\#\{i \mid x_i \ne c\}\\
    &=\argmax_c\#\{i \mid x_i=c\}\\
    &=\mathrm{mode}(\{x_i \mid i=1,\dots,n\})
  \end{align*}
  と分かる。
\end{proof}

\end{document}
