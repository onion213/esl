\documentclass[dvipdfmx,8pt]{beamer}
\usepackage{bxdpx-beamer} % dvipdfmxなので必要
\usepackage{pxjahyper} % 日本語のしおり用
% \usepackage{minijs} % フォントの設定?pLaTeX, upLaTaXなら不要

\usepackage{url} % 文中にリンク張る用
\usepackage{amsmath} %数式用
\renewcommand{\kanjifamilydefault}{\gtdefault} % 既定をゴシック体に変更


% \usetheme{metropolis}
\AtBeginSection{\frame{\sectionpage}} % Section毎に見出しを追加

\title{The Elements of Statistical Learning\\Chap.18: High-Dimensional Problems: $p \gg N$}
\date{\today}
\author{Kosuke Kito}

\begin{document}
  \maketitle
  \begin{frame}{流れ}
    \begin{itemize}
      \item 流れを書く。
    \end{itemize}
  \end{frame}
  \begin{frame}{今こそ再生核ヒルベルト空間}
    \begin{itemize}
      \item まずはやりたいことの整理。\\
        \begin{itemize}
          \item 線形に分類できないもの(同心円が典型例)をうまく分類したい。
          \item 変数変換してあげると、線形の手法を使える。\\
            \[
              (x,y) \mapsto (x^2,y^2)
            \]
          \item 高次元に写せば、ほぼ確実に線形分類可能にできる。
            \[
              (x,y)\mapsto (x,y,x^2,xy,y^2)
            \]
          \item 一方、高次元に写すと計算量が増えたり
        \end{itemize}
    \end{itemize}
  \end{frame}
\end{document}
